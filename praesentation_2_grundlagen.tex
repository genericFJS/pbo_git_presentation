\section{Grundlagen}
\begin{frame}
Hauptbefehle:
\begin{itemize}
	\item <1->init
	\note<1>{den Aktuellen Ordner zu einem Git Repository machen}
	\item <2->add
		\note<2>{Dateien/Änderungen zu einem aktuelle commit hinzufügen}
	\item <3->commit
		\note<3>{}
	\item <4->push
	\item <5->pull
	\item <6->merge
\end{itemize}
\end{frame}
\begin{frame}
	\includegraphics[scale=0.3]{Bilder/git_graph1}
	\note{Beispiel eines git Projektverlaufes}
\end{frame}
\begin{frame}
	\includegraphics[scale=0.3]{Bilder/git_graph2}
	\note{Graue Linie: Master Projekt mit aktuellsten Änderungen. Alles geht zurück auf das Master Projekt. Es existiert immer nur ein Projekt mit dem interagiert wird.}
\end{frame}
\begin{frame}
	\includegraphics[scale=0.3]{Bilder/git_graph3}
	\note{Bei einer Änderung wird ein indirekter Branch einer Projektes erstellt, wie zum Beispiel bei einem...(commit)}
\end{frame}
\begin{frame}
	\includegraphics[scale=0.3]{Bilder/git_graph4}
	\note{Commits sind eingetragene Änderungen. Diese Einträge sollten immer einen Titel haben der die Änderungen beschreiben. Dabei sollte man eine Struktur einhalten, mehr dazu später. Vorsicht: Nach dem eine Änderung commited wurde ist sie noch nicht im Master Projekt zu finden. Dafür müssen die Änderungen...(push)}
\end{frame}
\begin{frame}
	\includegraphics[scale=0.3]{Bilder/git_graph5}
	\note{Mit einem push leitet man den Vorgang ein, die noch temporären Änderungen nun einzutragen}
\end{frame}
\begin{frame}
	\includegraphics[scale=0.3]{Bilder/git_graph6}
	\note{Beim zusammenführen der Dateien zum Master Projekt wird ein merge gemacht, damit sind nun die Änderungen in das Master Projekt eingetragen worden}
\end{frame}
\subsection{Details}
\begin{frame}
Hier noch ein paar Details.
\end{frame}