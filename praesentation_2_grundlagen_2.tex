\subsection{gitignore}
\begin{frame}
\begin{itemize}
\item Blacklisting von Dateien/Ordnern
\begin{itemize}
\item Binäres \hfill \lstinline`*.exe`
\item Von Programmen \hfill \lstinline`.vscode/`
\item Geheimes \hfill \lstinline`my_password.txt`
\end{itemize}
\item Whitelisting
\begin{itemize}
\item Ausnahmen zur Blacklist \hfill \lstinline`!final/*`
\end{itemize}
\end{itemize}
\end{frame}

\begin{frame}[fragile]
\setstretch{1}
Datei \lstinline`.gitignore`:
\begin{lstlisting}[numbers=none, language=python]
# alle Binärdateien:
*.exe
*.out
# Ordner von VSC:
.vscode/
# Geheimes:
mein_passwort.txt
bildchen/
# Behalte Einzeldatei:
!bildchen/goodprof.png
\end{lstlisting}
\end{frame}

\begin{frame}[fragile]
\setstretch{1}
\begin{center}
\begin{minipage}{.3\textwidth}
\tiny
\begin{lstlisting}[numbers=none, language=python]
# alle Binärdateien:
*.exe
*.out
# Ordner von VSC:
.vscode/
# Geheimes:
mein_passwort.txt
bildchen/
# Behalte Einzeldatei:
!bildchen/goodprof.png
\end{lstlisting}
\end{minipage}
\qquad
\begin{minipage}{.3\textwidth}
%\scalebox{.65}{
\tiny
\begin{forest}
  for tree={
    font=\ttfamily,
    grow'=0,
    child anchor=west,
    parent anchor=south,
    anchor=west,
    calign=first,
    edge path={
      \noexpand\path [draw, \forestoption{edge}]
      (!u.south west) +(7.5pt,0) |- node[fill,inner sep=1.25pt] {} (.child anchor)\forestoption{edge label};
    },
    before typesetting nodes={
      if n=1
        {insert before={[,phantom]}}
        {}
    },
    fit=band,
    before computing xy={l=15pt},
  }
[repository/
	[{\color{htwgrey}.git/}]
  [programm/
  	[{\only<4->{\color{htwgrey}}\only<3>{\color{htworange}}.vscode/}
			[{\only<3->{\color{htwgrey}}…}]  	
  	]
    [readme.md]
    [virus.c]
    [{\only<2->{\color{htwgrey}}virus\_unix%
    {\only<2>{\color{htworange}}.out}}]
    [{\only<2->{\color{htwgrey}}virus\_win%
    {\only<2>{\color{htworange}}.exe}}]
  ]
  [{\only<5>{\color{htworange}}%
  \only<6>{\color{htwblue}}bildchen/}
    [{\only<5->{\color{htwgrey}}badprof.png}]
    [{\only<5-6>{\color{htwgrey}}\only<6>{\color{htwblue}}goodprof.png}]
  ]
  [.gitignore]
  [deren\_passwoerter.txt]
  [{\only<5->{\color{htwgrey}}\only<4>{\color{htworange}}mein\_passwort.txt}]
  [{\only<2->{\color{htwgrey}}no\_virus%
  {\only<2>{\color{htworange}}.exe}}]
  [virus.png]
]
\only<7>{}
\end{forest}
%}
\end{minipage}
\end{center}
\end{frame}

\subsection{Commit Nachricht Richtlinien}
\begin{frame}
Commit Nachricht besteht aus:
\begin{itemize}
\item Thema/Titel (\emph{subject})\\
Beschreibt Commit kurz.
\item Rumpf/Beschreibung (\emph{body})\\
Erläutert, \emph{was} \emph{warum} geändert wurde (nicht \emph{wie})
\end{itemize}
\end{frame}



\begin{frame}{Richtlinien Titel}
\begin{itemize}
\item Kurzer Titel (50, max. 72 Zeichen)
\item Mit Großbuchstabe beginnen
\item Kein Punkt am Ende des Titels
\item Imperativ benutzen
\note{Im Englischen: If applied, this commit will ...\\
Positiv: If applied, this commit will remove deprecated methods
Negativ: If applied, this commit will more fixes for broken stuff
}
\end{itemize}
\end{frame}

\begin{frame}{Negativbeispiel}
\footnotesize

\texttt{\noindent
Erweiterung Keylogger, der alles Aufgezeichnete in eine Datei im selben Ordner speichert.\bigskip\\
Es wurde ein Keylogger durch memory injection implementiert. In Zeilen 50-100 der Codedatei kann man gut sehen, wie die durch einen buffer overflow jedes Betriebssystem komprimiert werden kann.\\
Mehr Passwörter! Yay.
}
\end{frame}

\begin{frame}{Beispielnachricht}
\footnotesize

\texttt{\noindent
Erweitere Virus um Keylogger\bigskip\\
Es wurde ein Keylogger implementiert, der alle 
Nutzereingaben aufzeichnet und speichert.\\
Dadurch können mehr Passwörter ausgelesen werden.
}
\end{frame}


